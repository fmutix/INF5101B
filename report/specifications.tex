\section{Cahier des charges}

Le site de E-Commerce développé utilise comme base un restaurateur réel qui
vend des sandwich et des burgers. Notre équipe a décidé de créer un site
E-Commerce en tant que service supplémentaire que pourrait proposer ce
restaurateur. \\

Le site propose aux utilisateurs de remplir un panier de courses avec les
produits proposé par le restaurateur. Le site devra intégrer les
fonctionnalités suivantes :

\begin{itemize}
	\item Gérer des comptes utilisateur
	\item Gérer un panier de course dans la session utilisateur
	\item Propose un service de Livre d'or où les utilisateurs peuvent écrire
		des commentaires
	\item Implémenter un panneau d'administration
\end{itemize}

La gestion des comptes comprend une inscription des utilisateurs dans la base
de données à partir du site. Le site permet aussi aux utilisateurs de se
connecter et de se déconnecter. \\

Le panier de course n'est disponible qu'après la connexion d'un utilisateur. Il
contient les produits qui ont été ajoutés par ce dernier. Un utilisateur peut
ajouter, retirer des unités d'un produit ou supprimer complètement le produit
du panier. Une page de visualisation du panier est disponible où il est
possible de voir la liste des produits mis dans le panier avec leur quantité.
Le prix total de course est indiqué au bas du tableau de produits. \\

Le livre d'or est une page qui affiche tous les messages qui ont été laissé par
les utilisateurs. Seulement un utilisateur connecté pourra laisser un message.
A l'envoie du message, il sera enregistré dans la base de données avec la date
d'envoie, le message et son auteur. \\

Le panneau d'administation n'est accessible que par un compte avec un rôle
administrateur. Ce panneau donne accès aux produits et aux commandes des
clients présents dans la base de données. Les données sont présentées dans deux
pages. Une première pour les produits et une deuxième pour les commandes. \\

Sur la page des produits, tous les produits sont listés et plusieurs
fonctionnalités sont proposées. L'administrateur peut supprimer un produit de
la base de données ou au contraire en ajouter un nouveau à l'aide d'un
formulaire présent sur la page. \\

Pour la page des commandes, un historique de toutes les commandes y sont
représentées. Elles sont définies par le nom du client et la date de commande.
Sur chaque commande, il est possible de se rediriger vers une page plus
détaillée de la commande où le contenu du panier qui a été fait est déroulé.
