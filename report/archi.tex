\section{Description de l'architecture}

Le projet est divisé en 3 parties: les vues, les contrôleurs, et la logique
métier.\\

\section{Vues}

Les vues sont gérées par les Facelets de JSF. Celles-ci peuvent entre autrex
remplir les attributs contenus dans les ManagedBeans à l'aide de formulaires.

\section{Contrôleurs}

Les contrôleurs sont des Managed Bean JSF, ceux-ci font le lien entre les
Facelets et les entités. Par exemple la Managed Bean \hl{Account} possède une
instance de \hl{AccountEntity}. Cette instance sera remplie lorsqu'un visiteur
renseignera un des formulaires appropriés (i.e. loginForm, regstrationForm).
Ces classes contiennent également des méthodes renvoyant une String correspondant
au nom d'une page à afficher en fonction d'un événement (e.g. si un utilisateur
réussit à se connecter à son compte, il est redirigé vers la page d'accueil,
sinon le formulaire de connexion est rechargé avec un message d'erreur
personnalisé).\\

\begin{description}
    \item[Account] connexion, déconnexion, enregistrement d'un utilisateur
    \item[Food] récupère la liste de nourriture, et ajout de nouveaux éléments
    en passant par l'entité correspondante
    \item[GuestBook] envoi et supression de messages sur le livre d'or
    \item[Shop] envoi d'une commande
    \item[ShopAdm] gestion des pages d'administration d'historique de commandes
    et d'ajout de nourriture
\end{description}

\section{Modèles}

Les modèles sont des \hl{entit\'es}, c'est-à-dire des classes Java annotées afin
d'être persistées en base de données par un ORM (JPA ici). Le rôle de l'ORM est
de faire un lien entre l'application web qui manipule des objets, et la base de
données qui manipule des tables et des colonnes. JPA permet également de
construire des requêtes à partir d'objets et de méthodes, permettant ainsi de
réutiliser et chaîner celles-ci. En effet, sans cela les requêtes étaient
auparavant écrites directement sous forme de chaînes de caractères, ce qui est
peu maniable.\\

Notre cas d'utilisation étant relativement simple, nous n'avons pas eu besoin
de chaîner des requêtes. Nous avons cependant organisé le code de telle façon
qu'ajouter des requêtes complexes puisse se faire facilement.\\

A chaque entité est associé un \hl{Repository}, une classe contenant des méthodes
pour manipuler l'entité associée, et notamment des requêtes formées à partir
de JPA. Un \hl{Repository} est une \hl{LocalBean Stateless} afin d'avoir
accès à une annotation en particulier: \hl{@PersistenceContext}. Cette
annotation permet de gérer le cycle de vie d'un \hl{EntityManager} JPA
(création et destruction) automatiquement, pas besoin donc de créer une
\hl{EntityManagerFactory}.\\

\begin{description}
    \item[AccountEntity] compte utilisateur
    \item[FoodEntity] nourriture
    \item[MessageEntity] message du livre d'or
    \item[OrderEntity] commande
\end{description}
